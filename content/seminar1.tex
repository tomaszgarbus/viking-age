%ðÐþÞ%6789%1234%6789%1234%6789%1234%6789%1234%6789%1234%6789%1234%6789%1234%6789
\section{Seminar 1}

\subsection{Introducing the Viking Age}

\begin{itemize}
    \item background and context
    \item appearance of the "Viking"
    \item processes
    \item the sources to the viking age
\end{itemize}

\subsection{Viking}

\begin{itemize}
    \item mentioned in Norse scalding poetry 10th-11th
        century
    \item in E and S Scandinavia on rune stones mainly
        as personal name + a few instances as "occupation"
    \item \textit{wicings} in English texts (Widsið) which may be
        from end of 7th century (i.e. before the "Viking Age")
    \item \textit{wichingos} = pirates
    \item \textit{wic} -- vik (bay)
\end{itemize}

\subsection{Lindisfarne Monastery 793 AD}

\subsection{Battles of Stamford Bridge (York) and Hastings 1066}

\begin{itemize}
    \item End of the Viking Age, from the British perspective
\end{itemize}

\subsection{What and when was the Viking (Age)?}

\begin{itemize}
    \item A term for sea-faring, raiding Scandinavian men
    \item A label given by others to all Scandinavians
    \item A historical period (time-frame) c. 750-1100
\end{itemize}

Contemporary designations:

\begin{itemize}
    \item Rus' (rowers)
    \item Varangians (pledgers)
    \item Saqualibah (fair, ruddy)
    \item Fionn Gall (fair-haired strangers = Norse)
    \item Dubh Gall (dark-haired strangers = Danes)
    \item Lochlannach (people from the lakes/fjords)
    \item Danes, Northmen, heathens, pagans, ...
\end{itemize}

\subsection{6th-7th century: elite connections and shared material culture}

\begin{itemize}
    \item claw beakers (in Uppland, Gotland, also on British Isles)
    \item vendel ship burial, very similar to Sutton Hoo ship burial
    \item The Franks Casket -- Classic, Germanic, Jewish and Christian
        traditions: mythology, art styles, runes
\end{itemize}

\subsection{Proposed causes for the Viking Age}

\begin{itemize}
    \item warrior culture and ideal: honour
    \item quest for bridewealth: afford marriage
    \item population growth: settle elsewhere
    \item land inheritance rules: new opportunities for young ones
    \item other changes in home society: processes
\end{itemize}

\subsection{Processes: urbanisation}

\begin{itemize}
    \item wics (ports-of-trade, emporia)
    \item Dorestad (6th-9th C)
    \item Hedeby (Slesvig) (late 8th-1100)
    \item Ribe (c. 710-)
    \item Kaupang/Sciringsal (c. 750-900)
    \item Birka/Adelsö -- Sigtuna (8th C-)
    \item Ralswiek (8th C-)
    \item trading place, royal manor, sacred site, regular plots, coins
\end{itemize}

\subsection{Processes: Christianisation}

\begin{itemize}
    \item Harald Klak, king in Southern Denmark (Hedeby, Frisia)
    \item Baptized at Louis Pious' court AD 826
    \item Brought Ansgar to convert the Danes
    \item 829-31 Ansgar missionary to Birka in Sweden, protected by King Björn
    \item Harald Bluetooth (Denmark, c. 930-86) allowed missionaries and
        establishment of sees all through his reign, Baptised only in c. 960,
        Converted "all Denmark and Norway", remade his family monuments
        Christian (c. 965)
    \item 1070's Adam missionary accounts: Birka is in ruins, Uppsala is a
        heathen temple, Sigtuna is a Christian town
\end{itemize}

\subsection{Processes: state formation}

\begin{itemize}
    \item From Chiefs and Earls to the first kingdoms/states:
        Denmark (and Norway) 10th century, Sweden 12th century
    \item From external plunder to internal appropriation: from Gelds to Taxes
\end{itemize}

%1234%6789%1234%6789%1234%6789%1234%6789%1234%6789%1234%6789%1234%6789%1234%6789